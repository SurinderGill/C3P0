\documentclass[]{article}

% Imported Packages
%------------------------------------------------------------------------------
\usepackage{amssymb}
\usepackage{amstext}
\usepackage{amsthm}
\usepackage{amsmath}
\usepackage{enumerate}
\usepackage{fancyhdr}
\usepackage[margin=1in]{geometry}
\usepackage{graphicx}
\usepackage{extarrows}
\usepackage{setspace}
%------------------------------------------------------------------------------

% Header and Footer
%------------------------------------------------------------------------------
\pagestyle{plain}  
\renewcommand\headrulewidth{0.4pt}                                      
\renewcommand\footrulewidth{0.4pt}                                    
%------------------------------------------------------------------------------

% Title Details
%------------------------------------------------------------------------------
\title{Deliverable \#2 -- High-Level Architectural Design Document}
\author{SE 3A04: Software Design II -- Large System Design}
\author{Chen, Arthur \and Campbell, Christopher \and Endrizzi, Johnny \\ 
\and Coovert, Mitchell \and Gill, Surinder \and Dhadda, Terin}
\date{March 7, 2016}                              
%------------------------------------------------------------------------------

% Document
%------------------------------------------------------------------------------
\begin{document}

\maketitle	
\newpage
\tableofcontents
\listoftables
\newpage

% Introduction
%------------------------------------------------------------------------------
\section{Introduction}
\label{sec:introduction}
The following section provides a brief overview of the entire document.

\subsection{Purpose}
\label{sub:purpose}
The purpose of this document is to lay out the high level architectural design of the "BEER'D" application. It will first give a description of the system and a general overview of what it is for, how it is expected to be used, and why it is being developed. It also contains information about the variety of use cases for the application, an analysis class diagram, a breakdown of the intended architecture design, and finally a class responsibility collaboration breakdown. This document is intended primarily for the developers of the application, the professor, and the teaching assistants.

\subsection{System Description}
\label{sub:system_description}
The "BEER'D" system is a mobile application that aims to solve the question: "What beer is this?" This application is primarily being developed as a project for the third year Software Architecture class (course code SE 3A04) taught at McMaster University. A team of 6 students will design, develop, and create the application.\\
\\
The "BEER'D" application will take specific inputs from a user. Based on these inputs, varying "experts" will attempt to analyse and come up with their best prediction (based on data provided by publicly available API's) as to which beer the inputs may be identifying. The application will return and display a list of possible answers in a forum. Within this forum, users will also be able to share their answers on popular social media networks or find local stores which sell the beers referred to in the answers - based on their current location in an map.

\subsection{Overview}
\label{sub:overview}
The rest of the document is split up into four main sections:
\begin{enumerate}[-]
	\item The first section, Use Case Diagram, will contain each use case associated with the application. 
	\item The second section, Analysis Class Diagram, will contain the analysis class diagram for the application based upon the use case diagram.
	\item The third section, Architectural Design, will provide an overview of the overall architectural design for the application. It will first identify and provide reasoning for the chosen software architecture. Then, it explain the division of the system into subsystems and describe each subsystem.
	\item The fourth and final section, Class Responsibility Collaboration, will contain the "CRC Cards" of the application.
\end{enumerate}

% Use Case Diagram
%------------------------------------------------------------------------------
\section{Use Case Diagram}
\label{sec:use_case_diagram}
The following section provides a use case diagram for the application. 

\begin{enumerate}[a)]
	\item Each use case appearing in the diagram should be accompanied by a text description. 
\end{enumerate}


\section{Analysis Class Diagram}
\label{sec:analysis_class_diagram}
% Begin Section
This section should provide an analysis class diagram for your application.
% End Section


\section{Architectural Design}
\label{sec:architectural_design}
% Begin Section
This section should provide an overview of the overall architectural design of your application. You overall architecture should show the division of the system into subsystems with high cohesion and low coupling.

\subsection{System Architecture}
\label{sub:system_architecture}
% Begin SubSection
\begin{enumerate}[a)]
	\item Identify and explain the overall architecture of your system
	\item Be sure to clearly state the name of the architecture
	\item Provide the reasoning and justification of the choice
	\item Provide a structural architecture diagram showing the relationship among the subsystems (if appropriate)
\end{enumerate}
% End SubSection

\subsection{Subsystems}
\label{sub:subsystems}
% Begin SubSection
\begin{enumerate}[a)]
	\item Provide a brief description of each subsystem. Be sure to document its purpose and relationship to other subsystems.
\end{enumerate}
% End SubSection

% End Section
	
\section{Class Responsibility Collaboration (CRC) Cards}
\label{sec:class_responsibility_collaboration_crc_cards}
% Begin Section
This section should contain all of your CRC cards.

\begin{enumerate}[a)]
	\item Provide a CRC Card for each identified class
	\item Please use the format outlined in tutorial, i.e., 
	\begin{table}[ht]
		\centering
		\begin{tabular}{|p{5cm}|p{5cm}|}
		\hline 
		 \multicolumn{2}{|l|}{\textbf{Class Name:}} \\
		\hline
		\textbf{Responsibility:} & \textbf{Collaborators:} \\
		\hline
		\vspace{1in} & \\
		\hline
		\end{tabular}
	\end{table}
	
\end{enumerate}
% End Section

\appendix
% Begin Section
\newpage
\appendix
\section{Division of Labour}
\label{sec:division_of_labour}
\begin{table}[!htbp]
\centering
\begin{tabular}{|l|l|l|l|}
\hline
\multicolumn{1}{|c|}{\textbf{Team Member}} & \multicolumn{1}{c|}{\textbf{\begin{tabular}[c]{@{}c@{}}Student \\ Number\end{tabular}}} & \multicolumn{1}{c|}{\textbf{Contribution}} & \multicolumn{1}{c|}{\textbf{Signature}} \\ \hline
Arthur Chen & 1306616 &  &  \\ \hline
Christopher Campbell & 1143732 & &  \\ \hline
Johnny Endrizzi & 1310603 & &  \\ \hline
Mitchell Coovert & 1306701 & &  \\ \hline
Surinder Gill & 1308896 & &  \\ \hline
Terin Dhadda & 1312555 & Title, TOC, Introduction &  \\ \hline
\end{tabular}
\caption{Contributions and Signatures of Team Members}
\end{table}
% End Section
\newpage
\section*{IMPORTANT NOTES}
\begin{itemize}
%	\item You do \underline{NOT} need to provide a text explanation of each diagram; the diagram should speak for itself
	\item Please document any non-standard notations that you may have used
	\begin{itemize}
		\item \emph{Rule of Thumb}: if you feel there is any doubt surrounding the meaning of your notations, document them
	\end{itemize}
	\item Some diagrams may be difficult to fit into one page
	\begin{itemize}
		\item It is OK if the text is small but please ensure that it is readable when printed
		\item If you need to break a diagram onto multiple pages, please adopt a system of doing so and thoroughly explain how it can be reconnected from one page to the next; if you are unsure about this, please ask about it
	\end{itemize}
	\item Please submit the latest version of Deliverable 1 with Deliverable 2
	\begin{itemize}
		\item It does not have to be a freshly printed version; the latest marked version is OK
	\end{itemize}
	\item If you do \underline{NOT} have a Division of Labour sheet, your deliverable will \underline{NOT} be marked
\end{itemize}


\end{document}
%------------------------------------------------------------------------------