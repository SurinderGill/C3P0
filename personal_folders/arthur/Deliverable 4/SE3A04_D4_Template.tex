\documentclass[]{article}

% Imported Packages
%------------------------------------------------------------------------------
\usepackage{amssymb}
\usepackage{amstext}
\usepackage{amsthm}
\usepackage{amsmath}
\usepackage{enumerate}
\usepackage{fancyhdr}
\usepackage[margin=1in]{geometry}
\usepackage{graphicx}
\graphicspath{ {images/} }
\usepackage{extarrows}
\usepackage{setspace}
%------------------------------------------------------------------------------

% Header and Footer
%------------------------------------------------------------------------------
\pagestyle{plain}  
\renewcommand\headrulewidth{0.4pt}                                      
\renewcommand\footrulewidth{0.4pt}                                    
%------------------------------------------------------------------------------

% Title Details
%------------------------------------------------------------------------------
\title{Deliverable \#4 -- User Manual}
\author{SE 3A04: Software Design II -- Large System Design}
\author{Chen, Arthur \and Campbell, Christopher \and Endrizzi, Johnny \\ 
\and Coovert, Mitchell \and Gill, Surinder \and Dhadda, Terin}
\date {April 6, 2016}                              
%------------------------------------------------------------------------------

% Document
%------------------------------------------------------------------------------
\begin{document}

% Table of Contents
%------------------------------------------------------------------------------
\maketitle	
\newpage
\tableofcontents
\listoffigures
\listoftables
\newpage
%------------------------------------------------------------------------------

% Introduction
%------------------------------------------------------------------------------
\section{Introduction}
\label{sec:introduction}
This document outlines each page and its main functions as well as a step by step instruction on how to use the application.

\section{How To Use}
The primary method of operation of this application is to identify a beer by three different experts. To navigate to the results page:
\begin{enumerate}
\item Press the search button on the main page
\item Press on the arrow for each drop-down menu and select the options you want to search by.
\item The application will bring you to a Forum Page which guesses which beer you wanted to find.

\end{enumerate}

\subsection{Main Page}
\label{sec:main_page}
The main page consists of three buttons, the Search button, the Previous Search button, and the General Information button.
Each button will direct you to a new page.

\subsection{General Info Page}
\label{sec:general_info_page}
This page provides general information about beer and the BEER'D application. It also includes a breif description of how to use the application.

\subsection{Search Page}
\label{sec:search_page}
From this page the user can search for a beer of their choice. Using three drop down menus, corresponding to the type of beer, the style of beer, and the country it is from, a list of possible beers is curated. If you cannot find a specific input from the drop down menus, then you can leave the selection blank and our experts will attempt to find your beer regardless.

\subsection{Previous Search Page}
\label{sec:search_page}
If the BEER'D application has been previously used and has not been terminated on the device it is installed on, then this page will display the inputs used for the most recent search. After reviewing these inputs, they can be re-submitted and searched agin, to display the result previously seen. If the user has not used the BEER'D application to search during their current session, then this page will display blank results

\subsection{Forum Page}
\label{sec:forum_page}
The Forum page displays the results of your beer search. It also has buttons that will allow you to search again if the previous 
search was not correct. It also has a button to share the search result to Facebook.com if you want your friends to know about
a particular beer. The last button the allows you to open Google Maps and find the nearest LCBO to your current location. 

\section{Developer Information}
\label{sec:developer_information}
While altering the code asscociated with the BEER'D application, there are a few points to be aware of; the most important of these being: you should not run the Parser when the application is functioning. This may result in unwanted changes made to the appropriate text files, and ultimately lead to the application not functioning properly. The information gathered for the text files is in a JSON format, and therefore, the necessary frameworks are required to read the data. And finally, the data gathered for this application is from an LCBO (Liquor Control Board of Ontario) API, and may not be able to provide a full list of available beers to local users. Furthermore, the associated key is required to access the JSON data within the API, and is not provided as a part of this application.




% Division of Labour
%------------------------------------------------------------------------------
\newpage
\appendix
\section{Division of Labour}
\label{sec:division_of_labour}
\begin{table}[!htbp]
\centering
\begin{tabular}{|l|l|l|l|}
\hline
\multicolumn{1}{|c|}{\textbf{Team Member}} & \multicolumn{1}{c|}{\textbf{\begin{tabular}[c]{@{}c@{}}Student \\ Number\end{tabular}}} & \multicolumn{1}{c|}{\textbf{Contribution}} & \multicolumn{1}{c|}{\textbf{Signature}} \\ \hline
Arthur Chen & 1306616 & Google Maps Integration, User Manual &  \\ \hline
Christopher Campbell & 1143732 & Backend Coding, User Manual &  \\ \hline
Johnny Endrizzi & 1310603 & Facebook Integration &  \\ \hline
Mitchell Coovert & 1306701 & Google Maps Integration, User Manual &  \\ \hline
Surinder Gill & 1308896 & Application GUI, Backend Connection &  \\ \hline
Terin Dhadda & 1312555 & Backend Coding, Backend Connection &  \\ \hline
\end{tabular}
\caption{Contributions and Signatures of Team Members}
\end{table}
%------------------------------------------------------------------------------

\end{document}
%------------------------------------------------------------------------------
