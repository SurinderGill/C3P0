\documentclass[]{article}

% Imported Packages
%------------------------------------------------------------------------------
\usepackage{amssymb}
\usepackage{amstext}
\usepackage{amsthm}
\usepackage{amsmath}
\usepackage{enumerate}
\usepackage{fancyhdr}
\usepackage[margin=1in]{geometry}
\usepackage{graphicx}
\usepackage{extarrows}
\usepackage{setspace}
%------------------------------------------------------------------------------

% Header and Footer
%------------------------------------------------------------------------------
\pagestyle{plain}  
\renewcommand\headrulewidth{0.4pt}                                      
\renewcommand\footrulewidth{0.4pt}                                    
%------------------------------------------------------------------------------

% Title Details
%------------------------------------------------------------------------------
\title{Deliverable \#2 -- High-Level Architectural Design Document}
\author{SE 3A04: Software Design II -- Large System Design}
\date{}                               
%------------------------------------------------------------------------------

% Document
%------------------------------------------------------------------------------
\begin{document}

\maketitle	

\section{Introduction}
\label{sec:introduction}
% Begin Section

This section should provide an brief overview of the entire document.

\subsection{Purpose}
\label{sub:purpose}
% Begin SubSection
\begin{enumerate}[a)]
	\item Delineate the purpose of the document
	\item Specify the intended audience for the document
\end{enumerate}
% End SubSection

\subsection{System Description}
\label{sub:system_description}
% Begin SubSection
\begin{enumerate}[a)]
	\item Give a brief description of the system. This could be a paragraph or two to give some context to this document.
\end{enumerate}
% End SubSection

\subsection{Overview}
\label{sub:overview}
% Begin SubSection
\begin{enumerate}[a)]
	\item Describe what the rest of the document contains 
	\item Explain how the document is organised
\end{enumerate}
% End SubSection

% End Section

\section{Use Case Diagram}
\label{sec:use_case_diagram}
% Begin Section
This section should provide a use case diagram for your application. 
\begin{enumerate}[a)]
	\item Each use case appearing in the diagram should be accompanied by a text description. 
\end{enumerate}
% End Section

\section{Analysis Class Diagram}
\label{sec:analysis_class_diagram}
% Begin Section
This section should provide an analysis class diagram for your application.
% End Section


\section{Architectural Design}
\label{sec:architectural_design}
% Begin Section
This section should provide an overview of the overall architectural design of your application. You overall architecture should show the division of the system into subsystems with high cohesion and low coupling.

\subsection{System Architecture}
\label{sub:system_architecture}
% Begin SubSection
\begin{enumerate}[a)]
	\item Identify and explain the overall architecture of your system
	\item Be sure to clearly state the name of the architecture
	\item Provide the reasoning and justification of the choice
	\item Provide a structural architecture diagram showing the relationship among the subsystems (if appropriate)
\end{enumerate}
% End SubSection

\subsection{Subsystems}
\label{sub:subsystems}
% Begin SubSection
\begin{enumerate}[a)]
	\item Provide a brief description of each subsystem. Be sure to document its purpose and relationship to other subsystems.
\end{enumerate}
% End SubSection

% End Section
	
\section{Class Responsibility Collaboration (CRC) Cards}
\label{sec:class_responsibility_collaboration_crc_cards}
% Begin Section
This section should contain all of your CRC cards.

\begin{enumerate}[a)]
	\item Provide a CRC Card for each identified class
	\item Please use the format outlined in tutorial, i.e., 
	\begin{table}[ht]
		\centering
		\begin{tabular}{|p{5cm}|p{5cm}|}
		\hline 
		 \multicolumn{2}{|l|}{\textbf{Class Name:}} \\
		\hline
		\textbf{Responsibility:} & \textbf{Collaborators:} \\
		\hline
		\vspace{1in} & \\
		\hline
		\end{tabular}
	\end{table}
	
\end{enumerate}
% End Section

\appendix
% Begin Section
\newpage
\appendix
\section{Division of Labour}
\label{sec:division_of_labour}
\begin{table}[!htbp]
\centering
\begin{tabular}{|l|l|l|l|}
\hline
\multicolumn{1}{|c|}{\textbf{Team Member}} & \multicolumn{1}{c|}{\textbf{\begin{tabular}[c]{@{}c@{}}Student \\ Number\end{tabular}}} & \multicolumn{1}{c|}{\textbf{Contribution}} & \multicolumn{1}{c|}{\textbf{Signature}} \\ \hline
Arthur Chen & 1306616 &  &  \\ \hline
Christopher Campbell & 1143732 & &  \\ \hline
Johnny Endrizzi & 1310603 & &  \\ \hline
Mitchell Coovert & 1306701 & &  \\ \hline
Surinder Gill & 1308896 & &  \\ \hline
Terin Dhadda & 1312555 & &  \\ \hline
\end{tabular}
\caption{Contributions and Signatures of Team Members}
\end{table}
% End Section
\newpage
\section*{IMPORTANT NOTES}
\begin{itemize}
%	\item You do \underline{NOT} need to provide a text explanation of each diagram; the diagram should speak for itself
	\item Please document any non-standard notations that you may have used
	\begin{itemize}
		\item \emph{Rule of Thumb}: if you feel there is any doubt surrounding the meaning of your notations, document them
	\end{itemize}
	\item Some diagrams may be difficult to fit into one page
	\begin{itemize}
		\item It is OK if the text is small but please ensure that it is readable when printed
		\item If you need to break a diagram onto multiple pages, please adopt a system of doing so and thoroughly explain how it can be reconnected from one page to the next; if you are unsure about this, please ask about it
	\end{itemize}
	\item Please submit the latest version of Deliverable 1 with Deliverable 2
	\begin{itemize}
		\item It does not have to be a freshly printed version; the latest marked version is OK
	\end{itemize}
	\item If you do \underline{NOT} have a Division of Labour sheet, your deliverable will \underline{NOT} be marked
\end{itemize}


\end{document}
%------------------------------------------------------------------------------